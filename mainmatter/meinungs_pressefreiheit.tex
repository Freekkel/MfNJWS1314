\chapter{Grundgesetz, Grundrechte, Meinungsfreiheit, Pressefreiheit}
\section{Grundgesetz}
Das wichtigste Grundgesetz in diesem Zusammenhang der im Kapiteltitel genannten Bereiche ist Art 5 GG
\subsection{Art. 5 GG}

\textcolor{red}{(1) $^1$ Jeder hat das Recht, seine Meinung in Wort, Schrift und Bild frei zu
äußern und zu verbreiten und sich aus allgemein zugänglichen Quellen
ungehindert zu unterrichten. $^2$ Die Pressefreiheit und die Freiheit der
Berichterstattung durch Rundfunk und Film werden gewährleistet. $^3$ Eine
Zensur findet nicht statt.}

(2) Diese Rechte finden ihre Schranken in den Vorschriften der allgemeinen
Gesetze, den gesetzlichen Bestimmungen zum Schutz der Jugend und in
dem Recht der persönlichen Ehre.

\textcolor{red}{(3) $^1$ Kunst und Wissenschaft, Forschung und Lehre sind frei. $^2$  Die Freiheit der Lehre entbindet nicht von der Treue zur Verfassung.}
\qquad\\
\qquad\\
\qquad\\
Aufgrund diesem Artikel lässt sich folgendes Ableiten:\\
Kommunikationsfreiheiten Art. 5 Abs. 1 GG:
\begin{itemize}
         \item Meinungsfreiheit
         \item Informationsfreiheit
         \item Pressefreiheit
         \item Rundfunkfreiheit
         \item Filmfreiheit
\end{itemize}

Freiheiten des Art. 5 Abs. 3 GG:
\begin{itemize}
	\item Kunstfreiheit
	\item Freiheit der Wissenschaft, Forschung und Lehre
\end{itemize}

\section{Meinungsfreiheit}
Unter Meinungsfreiheit versteht man das Recht auf freie Meinungsäußerung. Jeder Mensch hat das Recht, seine Meinung frei und öffentlich kundzutun. Dies kann durch Wort, Schrift und Bild oder anderem Wege geschehen. Dadurch sind auch neue sich ständig ändernde Art der Äußerung geschützt. Jedoch gibt es zwei unterschiedliche Arten der Meinungsäußerungen, welche einen unterschiedlichen Schutz genießen.\\

\textbf{Werturteile/Meinungsäußerung} $\rightarrow$ stellungnehmende,dafürhaltende, meinende Äußerungen, auf deren Wert, Richtigkeit oder Vernünftigkeit es nicht ankommt. \\
Eine Meinungsäußerung ist eine subjektive Wertung oder Beurteilung weder einem Beweis, noch einer objektiven
Einordnung als „richtig“ oder „falsch“ zugänglich\\

\textbf{Tatsachenbehauptungen} $\rightarrow$ nur vom Schutzbereich umfasst, wenn sie Voraussetzung für das Bilden einer Meinung oder mit einem Werturteil des Behauptenden verbunden sind (was sehr häufig der Fall sein dürfte).\\
Eine Tatsachenbehauptung bezieht sich auf objektive Umstände in der Wirklichkeit, die (zumindest theoretisch) dem Beweis vor einem Gericht zugänglich sind, also etwa durch Urkunden, Zeugen oder Sachverständige bestätigt oder widerlegt werden können.\\

In der Vorlesung wurde die Frage gestellt, ob unter anderem folgende Aussage, eine Meinungsäußerung oder Tatsachenbehauptung sei. \\
\glqq{}Der Stoff ist giftig und krebserregend\grqq{}. \\
Dieser Satz ist eine Meinungsäußerung, da hier keine Angaben über die Dosierung oder Menge des Stoffes gemacht wurde und daher die Aussage nicht beweisbar ist. Andere Aussagen sind nicht so einfach einzuordnen, da es hier auch darauf ankommt, wer die Aussage tätigt. Nehmen wir mal an die Zeitung FAZ und Bild treffen jeweils die Aussage \glqq{}Person X in Steuerskandal verwickelt\grqq{}. Bei der Bild liegt die Vermutung nahe, dass es lediglich eine Meinungsäußerung ist. Bei der FAZ kann man eher von einer Tatsachenbehauptung ausgehen, da die Vergangenheit gezeigt hat, wie diese Presseerzeugnisse Ihre Artikel untermauern und berichten. Daher sind immer die Hintergründe zu prüfen und genau abzuwägen.\\

\textbf{Problem:} Schutzwürdigkeit der unwahren Tatsachenbehauptung (Lüge). \\

\subsection{Meinungsäußerungen sind grundsätzlich nicht angreifbar, aber:}
\begin{itemize}
    \item keine Beleidigungen
    \begin{itemize}
        \item Bezeichnung einer Person als „Schmarotzer“,
        \item Bezeichnung einer Aussage als „dummdreiste Lüge“
        \item Aussage, dass ein Richter „in Rente gehen solle, weil bei ihm der Kalk riesele“
    \end{itemize}
    \item keine Schmähkritik
    \begin{itemize}
        \item Wenn die Äußerung keinerlei sachlichen Bezugspunkt mehr hat und es nur noch um die Diffamierung oder Erniedrigung der Person als solcher geht.
    \end{itemize}
    \item keine unangemessene Herabwürdigung
        \begin{itemize}
            \item Eine unangemessene Herabwürdigung wurde bspw. angenommen, als der Ministerpräsident von RLP, Kurt Beck,
auf die Titelseite der Zeitschrift \glqq{}Titanic\grqq{} unter dem Titel \textcolor{red}{\glqq{}Problembär außer Rand und Band: Knallt die Bestie ab\grqq{}} abgebildet war.
        \end{itemize}
\end{itemize}
%
%
%
\subsection{Abgrenzung Meinungsfreiheit – Pressefreiheit}
\glqq{}Während die in einem Presseerzeugnis enthaltene Meinungsäußerung bereits durch Art. 5 Abs.1 Satz 1 GG geschützt ist, geht es bei der Garantie der Pressefreiheit um die einzelne Meinungsäußerung übersteigende Bedeutung der Presse für die individuelle und öffentliche Meinungsbildung.\grqq{}
(BVerfGE 85, 1 [12]).

\glqq{}Der Schutzbereich der Pressefreiheit ist daher berührt, wenn es um die im Pressewesen tätigen Personen in
Ausübung ihrer Funktion, um ein Presseerzeugnis selbst, um seine institutionell-organisatorischen Voraussetzungen und Rahmenbedingungen sowie um die Institution der freien Presse überhaupt geht.\grqq{}
(BVerfGE 85, 1 [12 f.]).
%
%
%
\subsection{Grundrechtsprüfung}
Es gibt 3 Prüfungsschritte, um zu ermitteln, ob die Aussage eine Schutzwürdige Aussage ist.
\begin{itemize}
    \item Schutzbereich des Grundrechts
        \begin{itemize}
            \item Persönlich und sachlich
        \end{itemize}
    \item Eingriff in den Schutzbereich
        \begin{itemize}
            \item  zielgerichteter oder faktischer Eingriff
        \end{itemize}
     \item Verfassungsrechtliche Rechtfertigung
        \begin{itemize}
            \item Bestimmung der Schranke
            \item Verhältnismäßigkeit der Einzelmaßnahme
        \end{itemize}
\end{itemize}
%
%
%
%
%
\section{Pressefreiheit}
Pressefreiheit bezeichnet das Recht von Rundfunk, Presse und anderen (etwa Online-) Medien auf freie Ausübung ihrer Tätigkeit, vor allem das unzensierte Veröffentlichen von Informationen und Meinungen. Die Pressefreiheit soll die freie Meinungsbildung gewährleisten. Die Pressefreiheit besitzt einen Rechtscharakter und ist lässt sich in die Individuelle und institutionelle Pressefreiheit unterteilen. 
\begin{itemize}
	\item Der Einzelne darf seine Tätigkeit ohne staatliche Beeinflussung ausüben.
	\item Freie Presse steht als Institut unter stattlichem Schutz
	\begin{itemize}
		\item Objektivrechtliche Dimension
		\item Presse gilt als Vermittler zwischen Volk und Staat und hat damit eine öffentliche Aufgabe und Kontrollfunktion
	\end{itemize}
\end{itemize}
\textbf{Schutzbereich:}
\begin{itemize}
	\item Wert gebundener Pressebegriff - nur solche Veröffentlichungen, die öffentliche Aufgabe gerecht werden? (systemwidrig - in dubio pro libertate)
	\item Periodisch erscheinende Druckwerke und Bücher
	\item Erzeugnisse der Buchdruckerpresse, alle zur Verbreitung geeigneten und bestimmtem Vervielfältigungen (z.B. Plakate)
	\item Weit und entwicklungsoffen
	\item Voraussetzung:  körperliches Trägermedium
\end{itemize}
Daraus ergibt sich das Problem, dass der Begriff Presse eine bestimmte Art der Verkörperung voraussetzt. Wenn man nun spiegel.de mit zdf.de versucht zu vergleichen, gibt es dass Problem der E-Presse. Da die Entwicklungsoffenheit ein wesentliches Kriterium der Presse und diese im Gegensatz zu Rundfunk und Film neutral des Verbreitungsweges sein muss, könnte man auf den Schluss kommen, dass E-Presse = Presse ist. Nun gibt es das Problem, das zdf.de und spiegel.de dasselbe Medium verwenden und dass das ZDF zum Rundfunk gehört (siehe ZDF-Staatsvertrag). Dennoch werden diese teilweise in unterschiedliche Kategorien eingeteilt. Daher hat sich die allgemeine Auffassung gebildet, dass der Verbreitungswegs entscheiden ist. Da es bei der E-Presse keine Verkörperung gibt, ähnlich wie beim Rundfunk, gilt die E-Presse als Rundfunk. \\
\newpage
\textbf{Was ist geschützt?}
\begin{itemize}
    \item Tätigkeiten von der Beschaffung der Informationen bis zur Verbreitung einer Nachricht
    \item Redaktionsgeheimnis
    \item Informantenschutz
    \item Tatsacheninformationen
    \item Meinungsäußerungen
\end{itemize}
\qquad\\
\textbf{Wer ist geschützt?}
\begin{itemize}
    \item Alle Personen, die produktiv, vermittelnd oder empfangend an der geistig-inhaltlichen Kommunikation durch die Presse teilnehmen
    \item  Herausgeber
    \item Redakteure
    \item Korrespondenten 
    \item freie Mitarbeiter 
    \item Verleger
    \item der einzelne Journalist
    \item Minderjährige (Schülerzeitungen)
    \item Juristische Personen des Privatrechts (Art. 19 Abs. 3 GG)
\end{itemize}
\qquad\\
\subsection{Eingriffe in die Pressefreiheit}
Eingriffe in die Pressefreiheit sind entweder \textbf{Final} oder \textbf{Mittelbar}. \\
\textbf{Beispiele für Final:}
\begin{itemize}
    \item Verbot der Berufsausübung als Redakteure
    \item Beschlagnahmen von Zeitungen
    \item Durchsuchung von Redaktionsräumen
    \item Einführung eines staatlichen Genehmigungsverfahren
\end{itemize}
\textbf{Beispiel für Mittelbar:}
Das Innenministerium des Landes NRW gibt jährlich Verfassungsschutzberichte zur Information der Öffentlichkeit heraus. Seit 1994 wird darin regelmäßig und ausführlich unter der Rubrik Rechtsextremismus u.a. über die Wochenzeitung „Junge Freiheit“ berichtet. So heißt es etwa, der Verfassungsschutz habe „zahlreiche tatsächliche Anhaltspunkte für den Verdacht rechtsextremistischer Bestrebungen.“ Es werden Warnungen oder ähnliches Verbreitet um Personen auf Gefahren aufmerksam zu machen. 
\subsection{Rechtfertigung eines Eingriffs in die Kommunikationsfreiheiten} 
\begin{itemize}
    \item Art. 5 Abs. 2 GG
    \item „Allgemeine Gesetze“ BVerfG: „Alle Gesetze, „die nicht eine Meinung als solche verbieten, die sich nicht gegen die Äußerung der Meinung als solche richten [Sonderrechtslehre], sondern dem Schutz eines schlechthin ohne Rücksicht auf eine bestimmte Meinung zu schützenden Rechtsguts dienen [Abwägungslehre].“
    \item Gesetzliche Bestimmungen zum Schutz der Jugend
    \item Recht der persönlichen Ehre
\end{itemize}
\textbf{Was zeichnet die Rechtfertigungsprüfung aus?}\\
Im Rahmen der Prüfung der Rechtfertigung eines Grundrechtseingriffs geht es stets um die \textbf{Abwägung} unterschiedlicher, jeweils von der Verfassung geschützter Rechtsgüter.

