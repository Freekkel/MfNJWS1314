\chapter{Allgemeines Persönlichkeitsrecht}
Das allgemeine Persönlichkeitsrecht ist wichtig, da es häufig als Schranke der Äußerungsfreiheit gilt. Dies ist Geregelt in Art. 2 Abs 1 in Verbindung mit Art. 1 Abs. 1 GG. \\
\section{Gewährleistung des Schutzes der Privatsphäre}
Dazu gehören alle Angelegenheiten, die typischerweise als privat eingestuft werden - räumlich und thematisch bestimmter Bereich, der grundsätzlich frei von unerwünschter Einsichtnahme bleiben soll. Dies ergibt sich aus Sphärentheorie. Diese Sphären sind Intimsphäre (umgangssprachlich: Schlafzimmer und alles was man dort macht) Privatsphäre (umgangssprachlich: Was man privat erledigt Zuhause oder bei Freundin in einem nicht öffentlichen Bereich) und die Sozialsphäre (umgangssprachlich: alles was auf der Arbeit passiert und in öffentlichen Bereichen).Jedoch gestaltet sich die Einordnung einer Handlung in eine Sphäre als problematisch. Je nach Fall ist eine Einteilung nicht möglich da die Übergänge fließend sind. \\
Weitere Rechte die sich aus dem allgemeinen Persönlichkeitsrecht sind:
\begin{itemize}
    \item Recht auf informationelle Selbstbestimmung
    \begin{itemize}
        \item Datenschutz (BDSG)
    \end{itemize}
    \item Recht am eigenen Wort
    \begin{itemize}
        \item geschriebenes Wort: Tagebücher (Dies kann zu Problemen führen, da man dadurch Tief in die Privat- oder Intimsphäre eindringt und dadurch unter Umsänden die Würde des Menschen verletzt.
        \item gesprochenes Wort: heimliche Tonbandaufnahmen. Diese führen in Regelmäßigen Abständen zu rechtlichen Problemen. 
    \end{itemize}
    \item Recht am eigenen Namen
    \begin{itemize}
        \item Anspruch auf Namensnennung § 13 UrhG
    \end{itemize}
    \item Recht am eigenen Bild
    \begin{itemize}
        \item Einwilligung § 22 KUG
    \end{itemize}
    \item Neue „Spielart“: Recht auf Gewährleistung der Vertraulichkeit und Integrität informationstechnischer Systeme – „IT-Grundrecht“
    \begin{itemize}
        \item "Online Durchsuchung"
        \item zum Schutz der freiheitlich demokratischen Grundordnung wurde der heimlichen Zugriff auf informationstechnische Systeme erlaubt.
    \end{itemize}
\end{itemize}
%
%
%
\section{Abbildungen ohne Zustimmungen (Übungsfall Coline)}
Abwägung zwischen Art. 8 EMRK (Schutz des Privatlebens) und Art. 10 EMRK (Meinungsfreiheit)
\begin{itemize}
    \item Figur der absoluten Person der Zeitgeschichte ist zu starr und unbestimmt für einen wirksamen Schutz des Privatlebens im Sinne des Art. 8 EMRK
    \item Der EGMR unterscheidet zwischen:
    \begin{enumerate}
        \item Politikern (politicians/personnes politiques) Schutz
        \item sonstigen im öffentlichen Leben oder im Blickpunkt der Öffentlichkeit stehenden Personen (public figures/personnes publiques) Schutz
        \item gewöhnlichen Privatperson (ordinary person/personne ordinaire) Schutz
    \end{enumerate}
    \item $\rightarrow$ Soweit Gruppe 2 und 3 betroffen, bei der Abwägung entscheidend:\\„public watchdog“ Funktion der Presse wird nur ausgelöst, wenn Beitrag zu einer Debatte von allgemeinem Interesse besteht
\end{itemize}
