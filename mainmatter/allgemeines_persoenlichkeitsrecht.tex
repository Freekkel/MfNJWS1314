\chapter{Allgemeines Persönlichkeitsrecht}
Das allgemeine Persönlichkeitsrecht ist wichtig, da es häufig als Schranke der Äußerungsfreiheit gilt. Dies ist Geregelt in Art. 2 Abs 1 in Verbindung mit Art. 1 Abs. 1 GG. \\
\section{Gewährleistung des Schutzes der Privatsphäre}
Dazu gehören alle Angelegenheiten, die typischerweise als privat eingestuft werden - räumlich und thematisch bestimmter Bereich, der grundsätzlich frei von unerwünschter Einsichtnahme bleiben soll. Dies ergibt sich aus Sphärentheorie. Diese Sphären sind Intimsphäre (umgangssprachlich: Schlafzimmer und alles was man dort macht) Privatsphäre (umgangssprachlich: Was man privat erledigt Zuhause oder bei Freundin in einem nicht öffentlichen Bereich) und die Sozialsphäre (umgangssprachlich: alles was auf der Arbeit passiert und in öffentlichen Bereichen).Jedoch gestaltet sich die Einordnung einer Handlung in eine Sphäre als problematisch. Je nach Fall ist eine Einteilung nicht möglich da die Übergänge fließend sind. \\
Weitere Rechte die sich aus dem allgemeinen Persönlichkeitsrecht sind:
\begin{itemize}
    \item Recht auf informationelle Selbstbestimmung
    \begin{itemize}
        \item Datenschutz (BDSG)
    \end{itemize}
    \item Recht am eigenen Wort
    \begin{itemize}
        \item geschriebenes Wort: Tagebücher (Dies kann zu Problemen führen, da man dadurch Tief in die Privat- oder Intimsphäre eindringt und dadurch unter Umständen die Würde des Menschen verletzt.
        \item gesprochenes Wort: heimliche Tonbandaufnahmen. Diese führen in Regelmäßigen Abständen zu rechtlichen Problemen. 
    \end{itemize}
    \item Recht am eigenen Namen
    \begin{itemize}
        \item Anspruch auf Namensnennung § 13 UrhG
    \end{itemize}
    \item Recht am eigenen Bild
    \begin{itemize}
        \item Einwilligung § 22 KUG
    \end{itemize}
    \item Neue „Spielart“: Recht auf Gewährleistung der Vertraulichkeit und Integrität informationstechnischer Systeme – „IT-Grundrecht“
    \begin{itemize}
        \item "Online Durchsuchung"
        \item zum Schutz der freiheitlich demokratischen Grundordnung wurde der heimlichen Zugriff auf informationstechnische Systeme erlaubt.
    \end{itemize}
\end{itemize}
%
%
%
\section{Abbildungen ohne Zustimmungen (Übungsfall Caroline)}
\subsection{Rechtsprechung des EGMR}
Abwägung zwischen Art. 8 EMRK (Schutz des Privatlebens) und Art. 10 EMRK (Meinungsfreiheit)
\begin{itemize}
    \item Figur der absoluten Person der Zeitgeschichte ist \underline{zu starr und unbestimmt} für einen wirksamen Schutz des Privatlebens im Sinne des Art. 8 EMRK
    \item Der EGMR unterscheidet zwischen:
    \begin{enumerate}
        \item Politikern (politicians/personnes politiques) Schutz
        \item sonstigen im öffentlichen Leben oder im Blickpunkt der Öffentlichkeit stehenden Personen (public figures/personnes publiques) Schutz
        \item gewöhnlichen Privatperson (ordinary person/personne ordinaire) Schutz
    \end{enumerate}
    \item $\rightarrow$ Soweit Gruppe 2 und 3 betroffen, bei der Abwägung entscheidend:\\„public watchdog“ Funktion der Presse wird nur ausgelöst, wenn Beitrag zu einer Debatte von allgemeinem Interesse besteht
\end{itemize}
%
%
%
\subsection{Reaktion der deutschen Rechtsprechung}
\begin{itemize}
    \item Aufgabe der Figur der absoluten Person der Zeitgeschichte
    \item Ob ein Bildnis der Zeitgeschichte (§ 23 I Nr. 1 KUG) vorliegt, bestimmt sich nach dem \\ \underline{Informationswert im konkreten Fall}
    \item Flexibler als EGMR: \textcolor{red}{auch unterhaltende Beiträge können einen Informationswert begründen}
    \item Presse muss Informationswert im Gerichtsprozess darlegen
    \item \underline{Anhaltspunkte} für die Abwägung:
    \begin{itemize}
        \item Bekanntheitsgrad der Person
        \item Zusammenhang mit einer Wortberichterstattung
        \item Vorverhalten der fotografierten Person
        \item Umstände der Anfertigung des Fotos
    \end{itemize}
    \item Bestätigung der Rechtsprechung. durch den EGMR
\end{itemize}

\section{Meinungs- /Pressefreiheit vs. Persönlichkeitsrecht (Übungsfall fixe Feder)}
\begin{enumerate}[label={\roman*)}]
    \item Schutzbereich
    \begin{itemize}
        \item Abgrenzung Meinungs- / Pressefreiheit (Art. 5 Abs.1 S.1 /Art. 5 Abs.1 S.2 GG)
        \begin{itemize}
            \item Die Pressefreiheit ist kein Spezialfall der Meinungsfreiheit
            \item Bei Wortberichterstattungen bleibt es beim Schutz durch die Meinungsfreiheit (so der EGMR, zunehmend deutlicher auch in den Entscheidungen des BVerfG)
            \item Pressefreiheit kommt in Betracht bei der Veröffentlichung von Bildern, beim Tendenzschutz, bei dem Schutz der Informationsbeschaffung, also vor allem bei pressetypischen Verhaltensweisen
        \end{itemize}
        \item \textbf{Persönlich}
        \begin{itemize}
            \item Jede Person, die die geschützte Tätigkeit ausübt, auch juristische Personen des Privatrechts
        \end{itemize}
        \item \textbf{Sachlich}
        \begin{itemize}
            \item Begriff der Meinung ist 
grundsätzlich weit zu verstehen. Erfasst sind Werturteile und Tatsachenbehauptungen, sofern sie zur Bildung von Meinungen beitragen können. 
            \begin{itemize}
                \item Nicht geschützt sind falsche Tatsachenbehauptungen sowie falsche Zitate
                \item Daraus ergibt sich folgendes Problem, ist ein Zitat eine Tatsachenbehauptung oder ein Werturteil?
            \end{itemize}
            \item Tatsachenbehauptung mit Meinungsbildrelevanz. Die Wiedergabe der ablehnenden Antwort war geeignet, zu einer Bewertung des Klägers beizutragen. 
        \end{itemize}
    \end{itemize}
    \item Eingriff (+)
    \begin{arrowlist} 
        \item Urteil hält die Veröffentlichung für rechtswidrig und spricht die Unterlassungsverpflichtung aus. 
    \end{arrowlist}
    \item Schranken
    \begin{itemize}
        \item \textbf{Allgemeine Gesetze}\\
        Im vorliegenden Fall § § 823, 1004 BGB in Verbindung mit dem allgemeinen Persönlichkeitsrecht (Art. 2 Abs. 1 i.V.m. Art. 1 Abs. 1 GG) 
    \end{itemize}
    \item Schranken-Schranken
    \begin{enumerate}[label=\arabic*.]
        \item \textbf{Verfassungsmäßigkeit der gesetzlichen Regelung} \\
        Hier keine Anhaltspunkte für verfassungsrechtliche Bedengen gegen §§ 823 Abs. 1, 1004 BGB
        \item \textbf{Verfassungsmäßigkeit der Einzelmaßnahme (Urteil)}\\
        Dies ist dann der Fall, wenn das Gericht die§§ 823 Abs. 1, 1004 BGB verfassungsgemäß angewendet hat. Insbesondere müsste der Grundsatz der Verhältnismäßigkeit gewahrt worden sein.
        \item \textcolor{red}{Wechselwirkungslehre beachten}
        \begin{enumerate}[label={\alph*)}]
            \item \textbf{Legitimer Zweck}\\
            Das Unterlassungsurteil dient dem Persönlichkeitsrechtsschutz des RA Emsig und damit einem verfassungsrechtlich legitimen Ziel.
            \item \textbf{Geeignetheit}\\
            Die Verurteilung zur Unterlassung der Veröffentlichung des Zitats ist auch ein taugliches Mittel zur Erreichung des Ziels.
            \item \textbf{Erforderlichkeit}\\
            Ein milderes Mittel, das gleichermaßen zur Zweckerreichung geeignet ist wie die Unterlassung der Zitatveröffentlichung ist nicht ersichtlich.
            \item \textbf{Angemessenheit (Verhältnismäßigkeit im eigentlichem Sinne)}\\
            Der Grundrechtseingriff und der mit dem Eingriff verfolgte Zweck müssen in einemabgewogenen Verhältnis zueinander stehen.
        \end{enumerate}
    \end{enumerate}
\end{enumerate}
Daher ist es in diesem Fall notwendig zu \textbf{Gewichten und Abwägen}. Bei der Abwägung it zunächst zu berücksichtigen, in welcher \textbf{Sphäre} das APR Emsigs (Anwalt) betroffen ist. 
\begin{itemize}
    \item Intimsphäre, Privatsphäre, Sozialsphäre
    \begin{itemize}
        \item Hier: \textbf{Sozialsphäre}, es geht um geschäftliche Kommunikation
    \end{itemize}
\end{itemize}
%
%
%
\subsection{Schutz des Selbstbildes?}
Das Grundrecht aus Art. 2 Abs. 1 i.V.m. Art. 1 Abs. 1 GG gewährt nach der ständigen Rechtsprechung des Bundesverfasungsgerichts seinem Träger keinen Anspruch darauf, öffentlich nur so dargestellt zu werden, wies seinem Selbstild entspricht und es ihm selbst genehm ist.\\
Wahre Tatsachen dürfen deshalb grundsätzlich verbreitet werden, aber kann dies tu einer \textbf{Unzulässigen Prangerwirkung} führen? 
Eine Anprangerung kann dazu führen, dass die regelmäßig zulässige Äußerung einer wahren Tatsache aus der Sozialsphäre im Einzelfall mit Rüksicht auf überwiegende Persönlichkeitbelange des Betroffenen zu untersagen ist. 
