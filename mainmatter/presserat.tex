\chapter{Presserat}
\begin{itemize}
    \item Freiwillige Selbstkontrolle der Printmedien
    \item Lobbyarbeit für die Pressefreiheit in Deutschland
    \item Bearbeiten von Beschwerden aus der Leserschaft
    \item Organisation ergibt sich aus Satzung
    \begin{itemize}
        \item Tragerverein mit folgenden Mitgliedern:Bundesverband Deutscher Zeitungsverleger (BDZV), Verband Deutscher Zeitschriftenverleger (VDZ), der Deutsche Journalisten-Verband (DJV) und die Deutsche Journalistinnen- und Journalisten-Union (dju) in ver.di 
    \end{itemize}
\end{itemize}
%
%
%
\section{Pressekodex}
Der Presserat hat die publizistischen Grundsätz in einem Pressekodex zusammengefasst. Darin finden sich REgeln für die tägliche Arbeit der Journaliten, die die Wahrung der journalistischen Berufsethik sicherstellen. so z.B.: 
\begin{itemize}
    \item Die Wahrung der Menschenwürde
    \item Grundliche und faire Recherche
    \item Klare Trennung von redationellem Text und Anzeigen
    \item Ergänzt werden die Grundsätze durch zusätzliche Richtlinien, die aufgrund aktuelle Enwticklungen und Ereignisse ständig fortgeschrieben werden. 
\end{itemize}
\textbf{In der Praxis sind folgende Relevant:}
\begin{itemize}
    \item \textcolor{red}{Ziffer 2 $\rightarrow$ Sorgfaltspflichten}
    \item \textcolor{red}{Ziffer 7 $\rightarrow$ Trennung von Werbung und Redaktion}
    \item \textcolor{red}{Ziffer 8 $\rightarrow$ Persönlichkeitsrechte, Identifizierung von Opfern}
    \item \textcolor{red}{Ziffer 11 $\rightarrow$ Sensationsberichterstattung}
\end{itemize}
%
%
%
\section{Beschwerdeverfahren}
\qquad\\
Maßgeblich hierfür ist die Bschwerdeordnung\\

\textbf{§ 1 – Beschwerdeberechtigung}\\
\\
(1) Jeder ist berechtigt, sich beim Deutschen Presserat allgemein über veröffentlichungen oder Vorgänge in der deutschen Presse zu beschweren. Beschwerde kann zudem einreichen, wer der Ansicht ist, dass die Verarbeitung von personenbezogenen Daten zu journalistisch redaktionellen Zwecken im Rahmen der Recherche oder Veröffentlichung das Recht auf Datenschutz verletzt.\\
\\
(2) Der Deutsche Presserat kann auch von sich aus ein Beschwerdeverfahreneinleiten
%
%
%
\section{Sanktionsmöglichkeiten}
\textbf{Der Presserat besitzt vier Sanktionsmöglichkeiten:}
\begin{enumerate}
    \item die öffentliche Rüge (mit Abdruckverpflichtung)
    \item die nicht-öffentliche Rüge (auf Abdruck wird verzichtet, z.B. aus Gründen des Opferschutzes)
    \item die Missbilligung
    \item den Hinweis
\end{enumerate}
