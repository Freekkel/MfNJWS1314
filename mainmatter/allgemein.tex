\chapter{Allgemeines}
\section{Material für die Klausur}
Folgende Unterlagen sind zugelassen für die Klausur und dürfen vorbereitet mitgebracht werden. Vorbereitet bedeutet, dass man mit Post-it’s (hier als Beispiel genannt) den Anfang eines Gesetzes und man wichtige Stellen innerhalb der Gesetze mit Textmarkern markieren darf. NICHT erlaubt sind sonstige Ergänzungen in digitaler oder handschriftlicher Form. 
%
\subsection{Bücher:}
{Fechner/Mayer (Hrsg.), Medienrecht,\\
Vorschriftensammlung, 9. Auflage\\
(2012/2013)
%
\subsection{Online Material zum Ausdrucken:}
\href{http://www.hans-bredow-institut.de/webfm_send/715}{Schulz (Hrsg.) Gesetzessammlung Information,\\
Kommunikation, Medien, 14. Auflage (2013)\\
Arbeitspapiere des Hans-Bredow-Instituts Nr. 16\\}
%
\subsection{Zusätzlich mitzubringen:}
Hinweisblatt Klausur in Ilias zu finden. \\
Ein leeres Blatt und einen Stift zum aufschreiben von Notizen während der Klausur.
%
%
%
\section{Was ist Medienrecht?}
Medienrecht ist eine Querschnittmaterie des öffentlichen Rechts, des Zivilrechts und des Strafrechts und beschäftigt sich mit den Regelungen privater und öffentlicher Information und Kommunikation. Diese Querschnittmenge der Rechtsgebiete bietet jedoch eine große Regelungslücke dank der sich schnell entwickelten Medien, welche erst im Nachhinein geregelt werden können. Das Medienrecht kann unterteilt werden in die inhaltsspezifischen Rechtsgebiete, Urheberrecht, Telekommunikationsrecht und das Rundfunkrecht. Die klassischen Gegenstände des Medienrechts sind Presse, Rundfunk, Multimedia und Internet. In dieser Vorlesung wurde besonders auf Urheberrecht, Presserecht und Telemedienrecht.
\begin{figure}[ht]
	\centering
	\includegraphics[width=10cm]{mainmatter/pics/bereiche.png}
	\caption[Medienrecht – was ist das?]{Medienrecht – was ist das? (Quelle: Dr. Ellen Wagner, Eva; Erster Foliensatz; Seite 5)} 
	\label{bereiche}
\end{figure}