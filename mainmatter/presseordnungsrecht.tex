\chapter{Presseordnungsrecht}
\textbf{Für den Pressebereich gilt der Grundsatz der Länderkompetenz!}\\
\textbf{Landespresse- bzw. Landesmediengesetze der Länder}\\

$\rightarrow$ Siehe auch Fechner: Mustergesetz mit anschließender Synopse\\
\begin{figure}[ht]
	\centering
	\includegraphics[width=10cm]{mainmatter/pics/presseordnung.png}
	\caption[Systematik des Landesmediengesetzes f. R.-P.]{Systematik des Landesmediengesetzes f. R.-P. (Quelle: Dr. Ellen Wagner, Eva; Folie 60)} 
	\label{LMG}
\end{figure}
\section {Kernbestimmungen}
\begin{itemize}
    \item Presse ist zulassungsfrei, § 4 Abs 2 LMG (§ 2 MusterG)
    \item Impressumgspflicht, § 9 Abs. 1, 2 LMG (§ 7 MusterG)
    \item Transparenzgebot, § 9 Abs. 4 LMG (§ 11 MusterG)
    \item Trennungsgrundsatz, § 13 LMG (§ 9 MusterG)
    \item Pflichtexemplar, § 14 LMG (§ 12 MusterG)
\end{itemize}
\section{Zulassungsfreiheit}
\textbf{§4 Abs. 2 LMG}\\
Die Tätigkeit der Medien [...] ist vorbehaltlich der nachfolgenden Bestimmungen und im Rahmen der Gesetze zulassungs- und anmeldefrei.\\
\\
Zum Vergleich:\\
\\
\textbf{§ 20 Abs.1 Satz 1 RStV (Rundfunkstaatsvertrag)}\\
Private Veranstalter bedürfen zur Veranstaltung von Rundfunk
einer Zulassung.
%
%
%
\section{Impressum}
\textbf{§9 Abs. 1 LMG}\\

Auf jedem in Rheinland-Pfalz erscheinenden Druckwerk (§ 3 LMG) müssen \textcolor{red}{Name oder Firma und Anschrift derjenigen Personen genannt sein, die das Werk gedruckt und verlegt haben,} beim Selbstverlag derjenigen Personen, die das Werk verfasst haben oder herausgeben.\\

[§ 3 Abs.2 Nr.1 LMG $\rightarrow$ Begriffsbestimmung zu Druckwerke, auch „...Texte in verfilmter oder elektronisch aufgezeichneter Form, besprochene Tonträger...“]\\

$\rightarrow$ Es soll Behörden und Dritten ermöglicht werden, die für den Inhalt eines Druckwerks Verantwortlichen jederzeit straf-, zivil- und presserechtlich haftbar zu machen
%
%
%
\section{Verantwortlichkeit}
\textbf{§9 Abs.2 LMG}\\

Auf den \textcolor{red}{periodischen Druckwerken (§ 3 Abs. 2 Nr. 2 LMG) sind ferner Name und Anschrift der redaktionell verantwortlichen Person anzugeben.} Sind mehrere für die Redaktion verantwortlich, so muss das Impressum Name und Anschrift aller redaktionell verantwortlichen Personen angeben; hierbei ist kenntlich zu machen, wer für welchen Teil oder sachlichen Bereich des Druckwerks verantwortlich ist.\\

Für den \textcolor{red}{Anzeigenteil} ist eine \textcolor{red}{verantwortliche Person} zu benennen; für diese gelten die Vorschriften über die redaktionell verantwortliche Person entsprechend.
%
%
%
\section{Strafbarkeit}
\textbf{§35 Abs.1 Nr. 3 LMG}\\
Mit Freiheitsstrafe bis zu einem Jahr oder mit Geldstrafe wird bestraft, wer als Person, die das Druckwerk verlegt, beim Selbstverlag das Werk verfasst oder herausgegeben hat, oder als redaktionell verantwortliche Person \textcolor{red}{in Kenntnis eines strafbaren Inhalts} des Druckwerks den Vorschriften über das Impressum nach \textcolor{red}{§ 9 Abs.1-5 zuwiderhandelt},\\
\[ \dots \]
%
%
%
\section{Auskunftsplicht}
\textbf{§9 Abs. 4 LMG $\rightarrow$ Auskunftspflicht}\\
Wer ein \textcolor{red}{periodisches Druckwerk verlegt, muss in der ersten Nummer eines jeden Kalenderhalbjahres im Druckwerk offen legen, wer an der Finanzierung des Unternehmens wirtschaftlich beteiligt ist}; bei Tageszeitungen ist bei Veränderungen der wirtschaftlichen Beteiligung dies zusätzlich in der nachfolgenden ersten Nummer jedes Kalendervierteljahres offen zu legen. \textcolor{red}{Wirtschaftlich beteiligt} im Sinne des Satzes 1 ist, wer mit \textcolor{red}{mehr als 5 v. H. am Kapital} beteiligt ist oder \textcolor{red}{über mehr als 5 v. H. der  Stimmrechte verfügt}. Für die nach Satz 1 offen zu legenden Angaben ist die Wiedergabe der aus dem Handelsregister und aus den zum Handelsregister eingereichten Schriftstücken zu entnehmenden Beteiligungsverhältnisse ausreichend.
%
%
%
\section{Kennzeichnungspflicht}
\textbf{§ 13 LMG}\\
Hat diejenige Person, die ein periodisches Druckwerk verlegt oder für den Anzeigenteil verantwortlich ist, für eine Veröffentlichung ein Entgelt erhalten, gefordert oder sich versprechen lassen, so ist diese Veröffentlichung, soweit sie nicht schon durch Anordnung und Gestaltung allgemein als Anzeige zu erkennen ist, \textcolor{red}{deutlich mit dem Wort "Anzeige" zu bezeichnen.} \\
(Siehe dazu, Übungsfall Bravo Folie 66 ff)
%
%
%
\section{Pflichtexemplar}
\textbf{§ 14 LMG}\\
(1) Von jedem Druckwerk, das in Rheinland-Pfalz verlegt wird, ist ohne Rücksicht auf die Art des Textträgers und das Vervielfältigungsverfahren von der Person, die das Druckwerk verlegt, unaufgefordert unmittelbar nach Beginn der Verbreitung unentgeltlich und auf eigene Kosten ein Stück (Pflichtexemplar) in marktüblicher Form an die von dem für das wissenschaftliche Bibliothekswesen zuständigen Ministerium bezeichnete Stelle abzuliefern. Satz 1 gilt nicht für
\begin{enumerate}
\item Druckwerke, die in einer geringeren Auflage als zehn Exemplare erscheinen, sofern es sich nicht um Druckwerke handelt, die einzeln auf Anforderung verlegt werden,
\item \lbrack \dots \rbrack
\end{enumerate}
%
%
%
\section{Zeugnisverweigerung}
\textbf{§ 53 Abs. 1 StPO}\\
\lbrack \dots \rbrack \\
Zur Verweigerung des Zeugnisses sind ferner berechtigt 
\\
5. Personen, die bei der Vorbereitung, Herstellung oder Verbreitung von Druckwerken \lbrack \dots \rbrack berufsmäßig mitwirken oder mitgewirkt haben.
\\
Die in Satz 1 Nr. 5 genannten Personen dürfen das Zeugnis verweigern über die Person des Verfassers oder Einsenders von Beiträgen und Unterlagen oder des sonstigen Informanten sowie über die ihnen im Hinblick auf ihre Tätigkeit gemachten Mitteilungen, über deren Inhalt sowie über den Inhalt selbst erarbeiteter Materialien und den Gegenstand berufsbezogener Wahrnehmungen. Dies gilt nur, soweit es sich um Beiträge, Unterlagen, Mitteilungen und Materialien für den redaktionellen Teil oder redaktionell aufbereitete Informations-und Kommunikationsdienste handelt.
%
%
%
\section{Verjährung}
\textbf{§ 37 LMG}\\
(1)Die Verfolgung von Straftaten nach diesem Gesetz oder von Straftaten, die mittels eines Druckwerkes oder durch die Verbreitung von Sendungen oder Angeboten strafbaren Inhalts begangen werden, verjährt bei Verbrechen in einem Jahr, bei Vergehen in sechs Monaten.\\

\lbrack \dots \rbrack\\

(3)Die Verjährung beginnt mit der Veröffentlichung oder
Verbreitung.\\

\textbf{Dagegen:}
„Normale“ strafrechtliche Verjährungfrist: mindestens 3 Jahre



