\chapter{Ansprüche gegen die Presse}
\begin{itemize}
    \item Der Gegendarstellungsanspruch
    \item Der Unterlassungsanspruch
    \item Die Berichtigungsansprüche
    \item Die Schadensersatzansprüche
\end{itemize}
%
%
%
\section{Gegendarstellungsanspruch}
\begin{itemize}
    \item v.a. in Landesressegesetzen oder Landesmediengesetzen normiert
    \item Abdruck der eigenen Sachverhaltsversion des vermeintlich Geschädigten
    \item Wahrheitsgehalt der Gegendarstellung wird nicht überprüft
    \item Verpflichtung zum Abdruck, es sei denn, am Abdruck besteht kein berechtigtes Interesse (Missbrauchsschutz)
    \item Grundsatz der waffengleichheit: Abdruck dort, wo die Erstmitteilung auch angesiedelt war
\end{itemize}

\subsection{§ 11 Abs. 1 LMG Rheinland-Pfalz}
Die \textcolor{red}{redaktionell verantwortliche Person} und die \textcolor{red}{Person, die ein periodisches Druckwerk verlegt}, sowie Rundfunkveranstalter sind \textcolor{red}{verpflichtet}, unverzüglich eine Gegendarstellung \textcolor{blue}{der Person oder Stelle}, die durch eine in dem Druckwerk oder der Rundfunksendung aufgestellte \textcolor{green}{Tatsachenbehauptung} \textcolor{blue}{betroffen} ist, \textcolor{red}{ohne Kosten} für die Betroffenen zum \textcolor{red}{Abdruck} zu bringen, zu verbreiten oder in das Angebot ohne Abrufentgelt aufzunehmen. Für die Wiedergabe einer Gegendarstellung zu einer im Anzeigen- oder Werbeteil verbreiteten Tatsachenbehauptung sind die üblichen Entgelte zu entrichten.\\
\qquad\\
\textbf{Anspruchsberechtigt:}
\begin{itemize}
    \item „Person“ = Natürliche und juristische Personen
    \item „Stelle“= Behörden, Körperschaften, Verbände, aber auch Regierungen, Bürgerinitiativen, nichtrechtsfähige Vereine und Gesellschaften
    \item Betroffenheit= Individuelle, unmittelbare oder mittelbare Berührung der eigenen Interessensphäre z.B.:
    \begin{itemize}
        \item Theaterintendant bei Berichten über Zustände im Theater
        \item Eltern bei Veröffentlichung über das minderjährige Kind im Haushalt
    \end{itemize}
\end{itemize}
\qquad\\
\textbf{Anspruchsberechtigt:}
\begin{itemize}
    \item Verantwortlicher Redakteur, nicht der Verfasser einer Meldung, Herausgeber, nicht zwingend der Chefredakteur
    \item Verleger, also derjenige, der die Vervielfältigung und Verbreitung im eigenen Namen durchführt, gleichgültig ob auf eigene oder fremde Rechnung
\end{itemize}
\textbf{Grundsatz:}\\
Bei der Gegendarstellung gilt das Prinzip der \textbf{Wahrheitsunabhängigkeit}\\
\textcolor{red}{\textbf{Ausnahme:}}\\
Das berchtigte Interesse fehlt bei \textcolor{red}{\textbf{offensichtlicher Unwharheit}} der Gegendarstellung\\
\textbf{Für die Offensichtlichkeit gelten aber strenge Maßstäbe:}\\
Die Unwahrheit muss für den unbefangenen Durchschnitssleser auf der Hand liegen und gerichtsbekannt sein. Daher sind mehrdeutige Aussagen schwer einzugliedern. 

\subsection{§ 11 Abs. 2 LMG Rheinland-Pfalz}
Die Gegendarstellung hat \textcolor{red}{ohne Einschaltungen und Weglassungen} in \textcolor{red}{gleicher Aufmachung} wie die Tatsachenbehauptung zu erfolgen. Bei Druckwerken muss sie in der nach Empfang der Einsendung \textcolor{blue}{nächstfolgenden} für den Druck nicht abgeschlossenen \textcolor{blue}{Nummer} in dem gleichen Teil des Druckwerks und mit \textcolor{red}{gleicher Schrift} wie der beanstandete Text abgedruckt werden; sie darf \textcolor{green}{nicht} in der Form eines \textcolor{green}{Leserbriefs} erscheinen. Eine \textcolor{purple}{Erwiderung} muss sich auf \textcolor{purple}{tatsächliche Angaben} beschränken; dies gilt bei periodischen Druckwerken nur, sofern die Erwiderung in derselben Folge oder Nummer erfolgt. Verbreitet ein Unternehmen der in § 3 Abs. 2 Nr. 1 Buchst. b oder c genannten Art eine Gegendarstellung, so ist die Gegendarstellung gleichfalls unverzüglich so weit zu veröffentlichen, wie die behauptete Tatsache übernommen wurde.

