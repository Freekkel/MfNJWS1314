\chapter{Ansprüche gegen die Presse}
\begin{itemize}
    \item Der Gegendarstellungsanspruch
    \item Der Unterlassungsanspruch
    \item Die Berichtigungsansprüche
    \item Die Schadensersatzansprüche
\end{itemize}
%
%
%
\section{Gegendarstellungsanspruch}
\begin{itemize}
    \item v.a. in Landespressegesetzen oder Landesmediengesetzen normiert
    \item Abdruck der eigenen Sachverhaltsversion des vermeintlich Geschädigten
    \item Wahrheitsgehalt der Gegendarstellung wird nicht überprüft
    \item Verpflichtung zum Abdruck, es sei denn, am Abdruck besteht kein berechtigtes Interesse (Missbrauchsschutz)
    \item Grundsatz der Waffengleichheit: Abdruck dort, wo die Erstmitteilung auch angesiedelt war
\end{itemize}

\subsection{§ 11 Abs. 1 LMG Rheinland-Pfalz}
Die \textcolor{red}{redaktionell verantwortliche Person} und die \textcolor{red}{Person, die ein periodisches Druckwerk verlegt}, sowie Rundfunkveranstalter sind \textcolor{red}{verpflichtet}, unverzüglich eine Gegendarstellung \textcolor{blue}{der Person oder Stelle}, die durch eine in dem Druckwerk oder der Rundfunksendung aufgestellte \textcolor{green}{Tatsachenbehauptung} \textcolor{blue}{betroffen} ist, \textcolor{red}{ohne Kosten} für die Betroffenen zum \textcolor{red}{Abdruck} zu bringen, zu verbreiten oder in das Angebot ohne Abrufentgelt aufzunehmen. Für die Wiedergabe einer Gegendarstellung zu einer im Anzeigen- oder Werbeteil verbreiteten Tatsachenbehauptung sind die üblichen Entgelte zu entrichten.\\
\qquad\\
\textbf{Anspruchsberechtigt:}
\begin{itemize}
    \item „Person“ = Natürliche und juristische Personen
    \item „Stelle“= Behörden, Körperschaften, Verbände, aber auch Regierungen, Bürgerinitiativen, nicht rechtsfähige Vereine und Gesellschaften
    \item Betroffenheit= Individuelle, unmittelbare oder mittelbare Berührung der eigenen Interessensphäre z.B.:
    \begin{itemize}
        \item Theaterintendant bei Berichten über Zustände im Theater
        \item Eltern bei Veröffentlichung über das minderjährige Kind im Haushalt
    \end{itemize}
\end{itemize}
\qquad\\
\textbf{Anspruchsberechtigt:}
\begin{itemize}
    \item Verantwortlicher Redakteur, nicht der Verfasser einer Meldung, Herausgeber, nicht zwingend der Chefredakteur
    \item Verleger, also derjenige, der die Vervielfältigung und Verbreitung im eigenen Namen durchführt, gleichgültig ob auf eigene oder fremde Rechnung
\end{itemize}
\textbf{Grundsatz:}\\
Bei der Gegendarstellung gilt das Prinzip der \textbf{Wahrheitsunabhängigkeit}\\
\textcolor{red}{\textbf{Ausnahme:}}\\
Das berechtigte Interesse fehlt bei \textcolor{red}{\textbf{offensichtlicher Unwahrheit}} der Gegendarstellung\\
\textbf{Für die Offensichtlichkeit gelten aber strenge Maßstäbe:}\\
Die Unwahrheit muss für den unbefangenen Durchschnittsleser auf der Hand liegen und gerichtsbekannt sein. Daher sind mehrdeutige Aussagen schwer einzugliedern. 

\subsection{§ 11 Abs. 2 LMG Rheinland-Pfalz}
Die Gegendarstellung hat \textcolor{red}{ohne Einschaltungen und Weglassungen} in \textcolor{red}{gleicher Aufmachung} wie die Tatsachenbehauptung zu erfolgen. Bei Druckwerken muss sie in der nach Empfang der Einsendung \textcolor{blue}{nächstfolgenden} für den Druck nicht abgeschlossenen \textcolor{blue}{Nummer} in dem gleichen Teil des Druckwerks und mit \textcolor{red}{gleicher Schrift} wie der beanstandete Text abgedruckt werden; sie darf \textcolor{green}{nicht} in der Form eines \textcolor{green}{Leserbriefs} erscheinen. Eine \textcolor{purple}{Erwiderung} muss sich auf \textcolor{purple}{tatsächliche Angaben} beschränken; dies gilt bei periodischen Druckwerken nur, sofern die Erwiderung in derselben Folge oder Nummer erfolgt. Verbreitet ein Unternehmen der in § 3 Abs. 2 Nr. 1 Buchst. b oder c genannten Art eine Gegendarstellung, so ist die Gegendarstellung gleichfalls unverzüglich so weit zu veröffentlichen, wie die behauptete Tatsache übernommen wurde.
\subsection{Merke zum Gegendarstellungsanspruch:}
\begin{itemize}
    \item Die Gegendarstellung richtet sich gegen \textbf{\begin{large}TATSACHENBEHAUPTUNGEN\end{large}}
    \item Prinzip der Wahrheitsunabhängigkeit
    \begin{itemize}
        \item In dem Verfügungsverfahren findet weder eine Prüfung der beanstandeten Behauptungen auf ihre Richtigkeit noch eine Prüfung der Richtigkeit der Behauptungen in der Gegendarstellung statt. 
     \end{itemize}
     \item Die herrschende Meinung in der Rechtsprechung wendet im Gegendarstellungsverfahren das so genannte \glqq{} Alles oder nichts Prinzip\grqq{} an
     \begin{itemize}
     \item Dies bedeutet, dass wenn nach Auffassung des Gerichts eine einzelne Formulierung unzulässig ist, dies dazu führen kann, dass der Anspruch auf Gegendarstellung insgesamt zurückgewiesen wird. 
     \end{itemize}
\end{itemize}
\textbf{Formelle Voraussetzungen für die Gegendarstellung}
\begin{itemize}
    \item Schriftlichkeit
    \item Druckreife
    \item Deutschsprachig bei deutschsprachiger Veröffentlichung. 
    \item Eigenhändige Unterzeichnung
\end{itemize}
\textbf{Übermittlung per Telefax möglich?}
\begin{itemize}
    \item Strittig!
\end{itemize}
\textbf{Gerichtliche Durchsetzung}
\begin{itemize}
    \item Verfahren auf Erlass einer einstweiligen Verfügung
    \item Kein Hauptsacheverfahren!
\end{itemize}
Bei jeder Gegendarstellung hat die Redaktion die Möglichkeit, etwas in ihrem Namen, klar abgegrenzt, zu vermerken. Dies kann zu einem enormen \glqq{} Redaktionsschwanz\grqq{} führen. Solche Anmerkungen können wie folgt lauten. \glqq{}Anmerkung der Redaktion: Nach dem sächsischen Pressegesetz sind wir verpflichtet, nicht nur wahre, sondern auch unwahre Gegendarstellungen abzudrucken.\grqq{}\\
%
%
%
%
%
\section{Unterlassungsanspruch}
\textbf{Wann kommt eine Unterlassungsanspruch in Betracht}
\begin{itemize}
    \item Unterlassung \textcolor{red}{unwahrer Tatsachenbehauptungen \\ $\rightarrow$ Setzt die Feststellung der Unwahrheit voraus !!} 
    \item Unterlassung von \textcolor{red}{Meinungsäußerungen/Werturteilen}, die die Grenzen zur Schmähkritik überschritten haben (auch Beleidigungen, unangemessene Herabwürdigungen)
    \item \underline{Aber auch:}Unterlassung von wahren Tatsachenbehauptungen
    \begin{itemize} 
        \item deren Verbreitung geeignet ist, den Betroffenen \glqq{}an den Pranger\grqq{} zu stellen
        \item aus der Intim- und Privatsphäre von Personen
        \item im Rahmen einer unbefugten Veröffentlichung und Weitergabe von Betriebsgeheimnissen
    \end{itemize}
\end{itemize}
\textbf{Unterlassungsanspruch}, analog § 1004 BGB.
\begin{itemize}
    \item \textbf{Ziel:} Abwehr künftiger Störungen!
    \item \textbf{Voraussetzungen des Anspruchs:}
    \begin{itemize}
        \item \textbf{Geschütztes Rechtsgut}
        \begin{itemize}
            \item Rechtsgüter d. § 
 Abs. 1 BGB
            \item Individualinteressen, die durch Schutzgesetze im Sinne des § 823 Abs. 2 BGB geschützt sind (Bsp. UWG)
        \end{itemize}
        \item \textbf{Drohender rechtswidriger Eingriff}
        \begin{itemize}
            \item Die Äußerung enthält erwiesenermaßen unwahre Tatsachen 
            \item Äußerung wahrer Tatsachen in bestimmten Fällen
            \item Meinungsäußerungen sind nur dann rechtswidrig, wenn die Grenze zur Schmähkritik überschritten ist!
        \end{itemize}
        \item \textbf{Begehungsgefahr}
        \begin{itemize}
            \item Wiederholungsgefahr (Unterlassungsanspruch nach Veröffentlichung)
            \item Erstbegehungsgefahr (Vorbeugender Unterlassungsanspruch)
        \end{itemize}
    \end{itemize}
\end{itemize}
\textcolor{red}{$\rightarrow$ Verschuldensunbhängige Haftung!}
%
%
%
\subsection{Begehungsgefahr}
\begin{itemize}
    \item \textbf{Wiederholungsgefahr} 
    \begin{itemize}
        \item Ein begangener rechtswidriger Eingriff gegründet grundsätzlich die Vermutung, dass auch in Zukunft gleichartige rechtswidrige Eingriffe bevorstehen. 
        \item Keine Wiederholungsgefahr besteht, wenn eine strafbewehrte Unterlassungserklärung abgegeben wird. 
    \end{itemize}
    \item \textbf{Erstbegehungsgefahr}
    \begin{itemize}
        \item Der Betroffene muss konkrete Umstände nachweisen, die eine Erstbegehungsgefahr belegen!
        \begin{itemize}
            \item Bei Film-und Bildaufnahmen hängt es davon ab, ob der redaktionelle Rahmen und die Rechtswidrigkeit des Einsatzes schon konkret bewertet werden kann. (in der Regel nicht!)
            \item Bsp. für Erstbegehungsgefahr: Jemand berühmt sich eines Rechts, eine bestimmte Veröffentlichung vornehmen zu dürfen.
        \end{itemize}
    \end{itemize}
\end{itemize}
%
%
%
\subsection{Umfang des Unterlassungsanspruchs}
\begin{itemize}
    \item Grundsätzlich: Kein Gesamtverbot eines Beitrags, rechtswidrige Inhalte müssen herausgestellt werden. 
    \item Bestimmtheitserfordernis ist zu beachten!
    \begin{itemize}
        \item Unterlassung \glqq{}sinngemäßer Äußerung\grqq{} zulässig, nicht dagegen das Verbot, \glqq{}den Eindruck zu erwecken, \dots\grqq{} oder das Begehren, alle Bildnisse \glqq{} aus dem privaten Alltag\grqq{} zu unterlassen.
    \end{itemize}
\end{itemize}
%
%
%
\subsection{Durchsetzung des Unterlassungsanspruchs}
Nicht zuletzt um Kosten im Sinne des § 93 ZPO (sofortiges Anerkenntnis) zu vermeiden, bedarf es zunächst einer \textcolor{red}{außergerichtlichen Abmahnung} mit folgendem Inhalt:
\begin{itemize}
    \item Benennung der konkreten Verletzungshandlung
    \item Forderung zur Abgabe einer strafbewehrten Unterlassungserklärung
    \item Fristsetzung zur Abgabe der strafbewehrten Unterlassungserklärung
    \item Androhung gerichtlicher Schritte bei fruchtlosem Ablauf der gesetzten Frist
\end{itemize}
\textbf{Besonderheit: Berichterstattung über eine Straftat}
\begin{itemize}
    \item Tat gehört(je nach Einzelfall, aber meistens) zum allgemeinen Zeitgeschehen, dessen Vermittlung Aufgabe der Medien ist.
    \item Interesse wird umso stärker, je mehr sich die Tat in Begehungsweise und Schwere von der gewöhnlichen Kriminalität abhebt
\end{itemize}
\textbf{Anzuerkennendes Interesse der Öffentlichkeit aus:}
\begin{itemize}
    \item Verletzung der Rechtsordnung
    \item Beeinträchtigung individueller Rechtsgüter
    \item Sympathie mit den Opfern
    \item Furcht vor Wiederholung
    \item Vorbeugung vor Straftaten
\end{itemize}
\textbf{Für aktueller Berichtserstattung gilt}
\begin{itemize}
    \item Wer den Rechtsfrieden bricht und durch diese Tat und ihre Folgen Mitmenschen angreift oder verletzt, muss sich nicht nur den hierfür verhängten strafrechtlichen Sanktionen beugen, sondern auch dulden, dass das von ihm selbst erregte Informationsinteresse der Öffentlichkeit auf den dafür üblichen Wegen befriedigt wird
    \item Je nach Schwere der Straftat gegebenenfalls Einschränkungen im Hinblick auf die Art und Weise der Darstellung
\end{itemize}
\textbf{Bis zeitlicher Distanz zur Tat}
\begin{itemize}
    \item Das Interesse des Täters, von einer Reaktualisierung seiner Verfehlung verschont zu bleiben, gewinnt zunehmende Bedeutung
    \item Interesse an einer Wiedereingliederung
    \item aber: auch die Verbüßung einer Strafhaft führt nicht dazu, dass ein Täter den uneingeschränkten Anspruch erwirbt, mit der Tat \glqq{}allein gelassen zu werden\grqq{}.
\end{itemize}
\textbf{Besonderheit Archivmittelungen}
\begin{itemize}
    \item Anerkennenswertes Interesse der Öffentlichkeit nicht nur an der Information über aktuelles Zeitgeschehen, sondern auch an der Möglichkeit, vergangene zeitgeschichtliche Ereignisse zu recherchieren
    \item Ein generelles Verbot der Einsehbarkeit und Recherchierbarkeit beziehungsweise. ein Gebot der Löschung aller früherer identifizierender Darstellungen in Onlinearchiven würde dazu führen, dass Geschichte getilgt und der Straftäter vollständig immunisiert würde
    \item Großer Aufwand für die Anbieter der Archive mit einem abschreckenden Effekt auf den Gebrauch der Meinungs- und Pressefreiheit
    \item Hierauf hat der Täter, insbesondere bei schweren Kapitalverbrechen, keinen Anspruch
\end{itemize}
\textbf{$\rightarrow$ Endergebnis: Kein Unterlassungsanspruch}
%
%
%
\section{Berichtigungsanspruch}
\begin{arrowlist}
    \item Rechtsgrundlage: § 1004 BGB analog (Folgenbeseitigungsanspruch)
    \item Ziel: Beseitigung einer fortdauernden Rechtsbeeinträchtigung durch eine Veröffentlichung
    \item Möglich:\\
    \textcolor{red}{Widerruf,\\ Richtigstellung,\\ Ergänzung}
\end{arrowlist}
\textbf{Arten der Berichtigung}
\begin{itemize}
    \item \textcolor{red}{Wiederruf}
    \begin{itemize}
        \item Der Anspruch Genommene erklärt, dass er die unwahre Äußerung \glqq{}wiedrruf\grqq{} oder dass die Äußerung \glqq{}unwahr\grqq{} ist. 
    \end{itemize}
    \item \textcolor{red}{Richtigstellung}
    \begin{itemize}
        \item Eine Richtigstellung erfolgt dann, wenn die beanstandete Behauptung zwar nicht in vollem Umfang zu widerrufen aber zumindest einzuschränken ist. Möglich auch: \textcolor{red}{Ergänzung}
    \end{itemize}
\end{itemize}
%
%
%
\subsection{Voraussetzungen des Berichtigungsanspruchs aus §1004 BGB analog:}
\begin{itemize}
    \item \textbf{Rechtsgutbeeinträchtigung durch unwahre Tatsachenbehauptung}
    \begin{itemize}
        \item Die Beweislast für die Unwahrheit der angegriffenen Tatsachenbehauptung liegt beim Anspruchsteller. 
    \end{itemize}
    \item \textbf{Fortdauer der Beeinträchtigung}
    \begin{itemize}
        \item BGH in Caroline von Monaco I: Auch nach einem Zeitraum von zwei Jahren kann bei einer auflagenstarken Zeitschrift noch eine verletzende Wirkung vorhanden sein.
    \end{itemize}
    \item \textbf{Rechtswidrigkeit}
    \begin{itemize}
        \item Verlangt wird nicht, dass der Störer rechtswidrig gehandelt hat; entscheidend ist, dass der Störzustand rechtswidrig ist. Beispiel: trotz erfolgter Wahrheitsprüfung kommt die Unwahrheit einer Meldung erst später ans Licht
        \item Dann: Richtigstellung
    \end{itemize}
    \item \textbf{Rechtsschutinteresse}
    \begin{itemize}
        \item Bsp. (-) bei freiwilliger redaktioneller Richtigstellung, bei einer Berichtigung von Nebensächlichkeiten
    \end{itemize}
    \item \textcolor{red}{Nicht: Verschulden}
\end{itemize}
%
%
%
\subsection{Berichtigung}
\begin{itemize}
    \item Form der Berichtigung
    \begin{itemize}
        \item Wie bei Gegendarstellungsanspruch
    \end{itemize}
    \item Anspruchsverpflichteter
    \begin{itemize}
        \item Derjenige, der die zu berichtigende Behauptung aufgestellt hat, ist auch zu ihrer Berichtigung verpflichtet. Soweit Behauptung nur (ohne sich zu Eigen machen) verbreitet wird, kann nur Distanzierung verlangt werden.
    \end{itemize}
    \item Durchsetzung des Berichtigungsanspruchs
    \begin{itemize}
        \item Der Berichtigungsanspruch kann – im Gegensatz zu Unterlassungsanspruch und Gegendarstellung – \textcolor{red}{nur} im Wege der \textcolor{red}{Hauptsacheklage} durchgesetzt werden.
    \end{itemize}
\end{itemize}
%
%
%
%
%
\section{Schadensersatzanspruch}
\begin{arrowlist}
    \item Ziel: Ausgleich von Vermögensnachteilen
    \item Rechtsgrundlage: § § 823, 824, 826 BGB
    \item Anspruchsvoraussetzungen:
\end{arrowlist}
\textbf{$\rightarrow$ Anspruchsvoraussetzungen:}
\begin{itemize}
    \item Rechtswidrige Verletzung geschützter Interessen
    \begin{itemize}
        \item Rechtswidriger Eingriff in das Persönlichkeitsrecht, Recht am Unternehmen (§823 Abs. 1 BGB) z. B. durch Schmähkritik
    \end{itemize}
    \item \textcolor{red}{Verschulden}
    \begin{itemize}
        \item § 276 BGB, Vorsatz und Fahrlässigkeit
    \end{itemize}
    \item Vermögensschaden
    \begin{itemize}
        \item Verglichen wird die aktuelle Vermögenslage durch das Schadensereignis mit der hypothetischen Vermögenslage, die ohne das Ereignis vorläge - Materielle Schäden! Auch: entgangener Gewinn, § 252 BGB
    \end{itemize}
    \item Kausalität
    \begin{itemize}
        \item Das Verhalten des Schädigers muss ursächlich sein für einen Schadenseintritt
        \item Die Verletzung muss ursächlich sein für die gesamte Schadenshöhe!
    \end{itemize}
\end{itemize}
\textcolor{violet}{Bilder von Personen, auch von solchen der Zeitgeschichte, dürfen ohne deren Zustimmung nicht wirtschaftlich genutzt werden.}\\
Sollen Bilder von Personen in der Werbung als Zugpferde des Unternehmens eingesetzt werden, sind sogenannte\\
\textbf{Testimonialverträge}\\
abzuschließen, die regelmäßig sehr differenzierte Klauseln zur Rechteübertragung beinhalten, etwa zu zeitlichen und örtlichen Grenzen der Rechteübertragung, der Art der Darstellung usw.\\
\subsection{Verwendung Bildern: Satire}
\textbf{Satire genießt zwar den Schutz der Meinungs- und Kunstfreiheit, aber \textcolor{violet}{die satirischen Elemente müssen}}
\begin{itemize}
    \item erkennbar sein
\end{itemize}
\textcolor{violet}{Die Satire darf}
\begin{itemize}
    \item \textbf{keine} unwahren Tatsachen verbreiten
    \item \textbf{keine} Schmähungen und Beleidigungen enthalten
    \item \textbf{keine} Rechte unbeteiligter Dritter verletzen 
    \item \textbf{nicht} neben der Werbung zurücktreten
    \item \textbf{nicht} nur dazu dienen, den Wirtschaftswert einer Marke oder eine Person auszubeuten $\rightarrow$ schwierige Abgrenzung im Einzelfall
\end{itemize}
\section{Entschädigungsanspruch}
\textbf{Anspruch auf Geldentschädigung}\\
Bei schweren Persönlichkeitsrechtsverletzungen hat der Verletzte neben den Ansprüchen auf Gegendarstellung, Unterlassung, Berichtigung und Schadenersatz einen Anspruch auf Geldentschädigung, § 823 BGB i. V. m. Art. 2 Abs. 1, 1 Abs. 1 GG.\\
\textbf{Voraussetzung der Geldentschädigung}\\
\begin{itemize}
    \item Schwere Persönlichkeitsrechtverletzung
    \begin{itemize}
        \item Verletzung der Intimsphäre
        \item Verletzung der Privatsphäre
        \item Erfundene Interviews und Zitate
        \item Schmähkritik
        \item Persönlichkeitsrechtsverletzung durch Einsatz in der Werbung (Herrenreiter-Fall)
    \end{itemize}
    \item Beeinträchtigung kann nicht in anderer Weise befriedigt werden.
    \begin{itemize}
        \item Berichtigung kann dazu führen, dass ein Geldentschädigungsanspruch ausscheidet.
        \item Höhe des Anspruchs: kann nach § 287 Abs. 1 ZPO vom Gericht \glqq{}nach freier Überzeugung\grqq{} bestimmt werden. Bei Verletzung Recht am eigenen Bild denkbar: Lizenzanalogie
    \end{itemize}
\end{itemize}
